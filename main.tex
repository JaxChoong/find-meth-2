\documentclass[a4paper, 12pt]{article}
\usepackage[american]{babel}
\usepackage[utf8]{inputenc}
\usepackage{csquotes}
\usepackage[style=apa,backend=biber,natbib=true]{biblatex}
\DeclareLanguageMapping{american}{american-apa}
\usepackage[T1]{fontenc}
\usepackage{mathptmx}
\usepackage{enumerate}
\usepackage{amsmath} 
\usepackage[margin=0.5in]{geometry}

\addbibresource{main.bib}

\renewcommand{\baselinestretch}{1.0}
\newcommand\nd{\textsuperscript{nd}\xspace}
\newcommand\rd{\textsuperscript{rd}\xspace}
\newcommand\nth{\textsuperscript{th}\xspace} 

\setlength\parindent{24pt}

% fill up your name, ID, contribution and paper title here
\author{
IZZMINHAL AKMAL BIN NORHISYAM \quad 242UC240JF \quad 25\% \\
CHOW YING TONG \quad 242UC244NK \quad 25\% \\
CHOONG JIA XUEN \quad 242UC244K1 \quad 25\% \\
LIM YU XUAN \quad 251UC18043 \quad 25\% \\
}
\title{ Research Proposal Title  }        

\begin{document}
\maketitle

\section*{Executive Summary}
This includes the problem statement, objectives, research methodology, expected output and significance of output in a summary form. 
\hfill
\\
\textbf{Keywords:} keyword1, keyword2 \\

\section{Introduction}
The field of artificial intelligence (AI) is experiencing a major boom, best captured by the exponential growth in the computational resources required to train the latest, top-notch models. According to \citet{Sevilla_Roldan_2024}, between February 2022 and May 2024, the amount of training compute experienced a 4.4x growth per year. The ever-increasing computational demand for training AI models goes hand in hand with the widespread adoption of AI in businesses spanning multiple industries. \citet{Maslej2025} found that 78\% of organizations reported using AI in at least one business function, increasing from 55\% in 2023 (p.~262). 

\par The increase in computing power behind modern AI relies on millions of physical servers, powered by massive data centers across the globe. Although these sophisticated infrastructures help make training and powering the newest AI models possible, they contribute significantly to the carbon footprint of the AI landscape. As \citet{Wu2022} point out in their research, embodied carbon, which refers to the emissions released during the manufacturing of hardware components, represents the largest share of AI’s overall carbon emissions (p.~5). 

\par Other than the manufacturing stage, another overlooked aspect when it comes to evaluating the carbon emissions of AI is the end-of-life phase for the same hardware components. Now, especially with the cycle of accelerated hardware obsolescence, the hardware devices that can no longer accommodate the increasing computational demands eventually results in electronic waste (e-waste). This problem poses a gaping hole that humanity has to overcome. Hence, we will attempt to investigate and quantify the scale of the e-waste generated by the AI industry.

\section{Problem Statement}
This section discusses the problem statement.. 

\section{Research Questions, Hypotheses and Objectives}
Minimum of two and maximum of four for each of the research questions, hypotheses, and research objectives. You can use numbering/bullet point to list them out.

\section{Literature Review}
This section contains analytic discussion of the related research papers. 

\section{Research Methodology}
This section contains a description about the steps involved in research methodology, which also includes the metrics to be used for evaluating the proposed method along with the brief description of the techniques/models/methods/algorithms to be used.

You can use table/figure/image/graph in the report. All figures must have titles.

\section{Research Activities and Milestones}
Use a flowchart for the research activities and a Gantt chart for the research schedules. 

\section{Expected Results and Impact}
In this section, discussion will be on novel theories/findings/knowledge and the impact on society, nation and/or economy.

%References
\printbibliography

\end{document}

