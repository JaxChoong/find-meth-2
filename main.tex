\documentclass[a4paper, 12pt]{article}
\usepackage[american]{babel}
\usepackage[utf8]{inputenc}
\usepackage{csquotes}
\usepackage[style=apa,backend=biber,natbib=true]{biblatex}
\DeclareLanguageMapping{american}{american-apa}
\usepackage[T1]{fontenc}
\usepackage{mathptmx}
\usepackage{enumerate}
\usepackage{amsmath} 
\usepackage[margin=0.5in]{geometry}

\addbibresource{main.bib}

\renewcommand{\baselinestretch}{1.0}
\newcommand\nd{\textsuperscript{nd}\xspace}
\newcommand\rd{\textsuperscript{rd}\xspace}
\newcommand\nth{\textsuperscript{th}\xspace} 

\setlength\parindent{24pt}

% fill up your name, ID, contribution and paper title here
\author{
IZZMINHAL AKMAL BIN NORHISYAM \quad 242UC240JF \quad 25\% \\
CHOW YING TONG \quad 242UC244NK \quad 25\% \\
CHOONG JIA XUEN \quad 242UC244K1 \quad 25\% \\
LIM YU XUAN \quad 251UC18043 \quad 25\% \\
}
\title{ Research Proposal Title  }        

\begin{document}
\maketitle

\section*{Executive Summary}
This includes the problem statement, objectives, research methodology, expected output and significance of output in a summary form. 
\hfill
\\
\textbf{Keywords:} keyword1, keyword2 \\

\section{Introduction}
The field of artificial intelligence (AI) is experiencing a major boom, best captured by the exponential growth in the computational resources required to train the latest, top-notch models. \citet{Sevilla_Roldan_2024} report that the amount of training compute increased by 4.4 times annually between February 2022 and May 2024. The widespread use of AI in companies across various industries is accompanied by an ever-increasing computational demand for training AI models. According to \citet{Maslej2025}, 78\% of businesses used AI in at least one business function, up from 55\% in 2023 (p.~262).

\par Modern AI's increased processing capacity depends on millions of physical servers that are driven by enormous data centers located all over the world.  These advanced infrastructures greatly increase the carbon footprint of the AI landscape, despite helping to enable training and powering the most recent AI models.  Embodied carbon, or the emissions released during the manufacturing of hardware components, accounts for the largest portion of AI's total carbon emissions, as noted by \citet{Wu2022} in their research (p.~5).

\par Other than the manufacturing stage, another overlooked aspect when it comes to evaluating the carbon emissions of AI is the end-of-life phase for the same hardware components. Nowadays, especially with the cycle of accelerated hardware obsolescence, the hardware devices that can no longer accommodate the increasing computational demands eventually results in electronic waste (e-waste). 

\section{Problem Statement}
\subsection*{The Root Problem}
The obsolescence of hardware that are no longer able to meet the computational demands of AI creates a global e-waste crisis, which is not adequately addressed by existing research and current industry practices. The scale of the problem, and the potential solutions to it are still unclear.

\subsection*{Research Gaps}
Through analyzing multiple research works, several research gaps can be identified: 
\begin{enumerate} 
	\item Lack of Long-Term E-Waste Forecasting: Although \citet{wang_2024_ewaste} present a foundational model that forecasts e-waste, its scope is limited to only the year 2030. Infrastructure policy planning and developing standard industry practices for dealing with obsolete devices are usually long-term efforts, a forecast that extends to at least 2035 is necessary. 
	
	\item Ethical and Regulatory Gaps: As highlighted by \citet{Zhuk2023}, there is an ethical and legal grey area when it comes to handling e-waste. The toxic components of e-waste from developed countries usually end up in the landfills of developing nations that do not have the capacity or recycling infrastructure to dispose of the e-waste properly. There is a need to revamp existing disposal policies to address this issue.
	
	\item Absence of Quantified Solutions: Although \citet{wang_2024_ewaste} found circular economy strategies for hardware devices to be useful in reducing e-waste by 58\%, the precise implementation methods for circular economy is unclear and not well-documented.
\end{enumerate}

\subsection*{Analysis and Comparison of Possible Research Problems}
Based on the research gaps identified, there are several research problems that arise: 

\begin{enumerate}
    \item The Strategic Planning Problem: Without accurate, long-term projections of AI-driven e-waste after 2030, governments and businesses are unable to make strategic, well-informed decisions regarding the development of sustainable hardware, the allocation of resources for waste management, and future recycling infrastructure. As a result, society is not ready for the full impact of the e-waste wave that is anticipated to occur in the next ten to fifteen years.
 
	\item The Environmental Injustice Problem: There is a serious environmental injustice issue as a result of the lack of clear international regulations and moral standards for the disposal of AI e-waste.  The harmful effects of the AI revolution are disproportionately felt by developing countries, which have not profited from the advancement of the technology and are now dealing with chronic soil contamination, water pollution, and public health emergencies.
	
    \item The Implementation Issue: Although the AI sector is aware of the concepts of circular economy, there is a serious implementation issue.  Businesses are reluctant to invest in strategies like hardware lifespan extension or module reuse because there are no standardized models or metrics to assess their efficacy and economic feasibility. This inaction perpetuates the linear "take-make-dispose" model, directly contributing to the growing e-waste crisis.
\end{enumerate}

Despite the enormous global significance of problem (2), solving the fundamental problems of international law and trade policy would necessitate a scope that goes well beyond this computer science research project.  Similarly, the highly proprietary nature of hardware life cycle data and the absence of standardized measurement models hinder a thorough investigation, even though Problem (3) is crucial for the AI landscape.

\subsection*{Selection of A Research Problem}
After careful consideration, Problem (1) is selected as the main focus for this research proposal for the reasons as follows: 

\begin{itemize}
	\item Relevance and Significance: An essential first step is a forecast that lasts until at least 2035. It offers the information required to spur action on the other two issues. It is challenging to generate political will for new regulations (Problem 2) or encourage industry investment in innovative solutions (Problem 3) without a clear understanding of the actual scope of the upcoming crisis.
	
	\item Feasibility: This research can directly expand on the Computational Power-driven Material Flow Analysis (CP-MFA) model that Wang et al. (2024) have already proposed. With the methodology already in place, this research can concentrate on increasing its predictive capacity by adding fresh data and honing the hypotheses over a longer period of time.
	
	\item Measurable Outcome: A precise, quantifiable, and useful forecast will be the research's output.  Policymakers, business executives, and other researchers in the field will find this concrete outcome to be a useful contribution.
\end{itemize}

\section{Research Questions, Hypotheses and Objectives}
Minimum of two and maximum of four for each of the research questions, hypotheses, and research objectives. You can use numbering/bullet point to list them out.

\section{Literature Review}
This section contains analytic discussion of the related research papers. 

\section{Research Methodology}
This study employs a dynamic, probabilistic modeling approach to analyze the electronic waste (E-waste) generated by generative artificial intelligence (GAI) infrastructure, specifically AI servers supporting large language models (LLMs). Building on the Computational Power-driven Material Flow Analysis (CP-MFA) framework introduced by Wang et al. (2024), the methodology incorporates several improvements to realistically represent hardware lifecycle, operational factors, technology adoption, and circular economy impacts.

\subsection{Model Structure}
The core of the model predicts quarterly global deployments, in-use stocks, and end-of-service (EoS) E-waste from AI servers from 2020 to 2035. The computational power demand is linked to a benchmark GPU server (8-unit Nvidia DGX H100 system) to convert computational requirements into physical hardware stock and waste. Unlike prior exponential growth assumptions, this method applies a sigmoid adoption curve reflecting training-data-size constraints and technology maturation.

\subsection{Key Parameterization}
The model parameters are informed by empirical data and literature:

\begin{enumerate}
    \item Server Lifespan: Modeled as a normal distribution with a mean of 3.5 years and standard deviation of 0.8 years, reflecting industry data for compute hardware lifespans \citep{wang_2024_ewaste}.
    \item GPU Efficiency Gains: Represented by a uniform distribution between 15\% and 40\% efficiency improvement per generation, informed by Nvidia GPU power management studies showing generation-over-generation reductions in power use \citep{nvidia-2023, you-2023}.
    \item Data Center Uptime: Modeled as a Beta distribution with shape parameters $\alpha=17$ and $\beta=1.5$, ranging from 85\% to 98\%, capturing realistic enterprise data center operational uptimes across Tier 1 to Tier 4 facilities \citep{unknown-author-no-date, operations-2024}.
    \item User Adoption Rates: Follow a log-normal distribution consistent with technology diffusion literature and sigmoid S-curve adoption behavior observed in AI deployments \citep{operations-2024, pamplona-2024}.
    \item Market Penetration Caps: Triangular distribution with minimum 60\%, mode 80\%, and maximum 95\%, capturing saturation and market maturity levels derived from technology adoption frameworks \citep{fordyce-2025}.
    \item Hardware Failure Rates: Beta distribution ranging 2\%-8\% annually, based on reported failure rates of GPUs and server components in operational data centers \citep{wang_2024_ewaste}.
    \item Utilization Rates: Normal distribution with mean 70\% and standard deviation 15\%, truncated between 40\% and 90\%, consistent with reported server workload levels \citep{wang_2024_ewaste}.
    \item Regional Deployment Timing: Applied as a normal distribution with ±6 months variation to model geographic variability in AI infrastructure rollouts.
    \item Upgrade Thresholds: Uniform distribution spanning 20\%-40\% performance degradation due to factors like thermal throttling and aging hardware, motivating hardware refresh decisions \citep{you-2023}.
    \item Recycling Success Rates: Beta distribution with 70\%-95\% material recovery rates, based on current e-waste recycling performance worldwide \citep{wang_2024_ewaste}.
\end{enumerate}

\subsection{Methodological Enhancements}
\subsubsection{Dynamic GPU Power Consumption Modeling}
Modern AI hardware exhibits significant variability in power consumption over time due to workload demands, thermal management, and architectural improvements. Instead of assuming constant computational power intensity per GPU server, the model integrates dynamic power consumption curves that capture real-world variations. Empirical studies from Nvidia GPU power management demonstrate generation-over-generation power efficiency gains ranging from 15\% to 40\%, achieved through advanced power sharing, dynamic voltage and frequency scaling (DVFS), and task-specific optimization. The model also accounts for thermal throttling and gradual performance degradation over the lifespan of GPUs, which typically triggers upgrades once performance drops between 20\% to 40\%. Incorporating these time-varying and hardware-specific behaviors enhances the accuracy of power usage estimation and aligns e-waste projections with observed operational realities in AI data centers \citep{you-2023, nvidia-2023, wang_2024_ewaste}.

\subsubsection{Data Center Uptime Integration}
Operational uptime in data centers is never absolute; factors like scheduled maintenance, unexpected outages, and load balancing create realistic periods of server inactivity or reduced utilization. The model applies a Beta distribution to represent enterprise data center uptime percentages, spanning 85\% to 98\%, reflecting the diversity from Tier 1 to Tier 4 facility standards. This integration adjusts server utilization rates downward from naive 100\% assumptions, realistically reflecting operational availability that impacts effective computational capacity and hardware wear. Such nuanced uptime modeling ensures that the estimation of deployed server stock and resulting e-waste properly considers real operational environments and maintenance cycles, ultimately leading to more precise forecasting of EoS hardware volumes \citep{unknown-author-no-date, operations-2024, wang_2024_ewaste}.

\subsubsection{Sigmoid Technology Adoption Curve}
Classical models rely on exponential computational growth via Moore’s Law, which simplifies technology proliferation estimates but fails to accommodate the real-world saturation effects of market limits and technology maturation. This method replaces exponential growth with a sigmoid or S-curve adoption trajectory that captures the initial slow uptake, rapid mid-phase expansion, and eventual plateauing as the market saturates the addressable user base and computational demands stabilize. The log-normal distribution fits historical technology adoption patterns and reflects variability in global AI infrastructure deployments. Embracing sigmoid growth constraints tied to a finite training data set and realistic adoption limits generates plausible scenarios for GAI proliferation and associated e-waste flows more consistent with observed technology diffusion dynamics \citep{pamplona-2024, fordyce-2025, wang_2024_ewaste}.

\subsubsection{Heterogeneous Hardware Modeling}
Data centers deploy a mixed inventory of server hardware encompassing varying generations of GPUs, CPUs, and emergent AI accelerators like TPUs or custom ASICs. The enhanced model captures this heterogeneity by representing servers with differing performance and power profiles and modeling staggered, non-uniform hardware refresh cycles. For example, GPU upgrades often occur approximately every two years, aligning with commercial release and deprecation schedules. This approach departs from simplified uniform-replacement assumptions to reflect realistic operational practices, including partial fleet refresh and maintenance-driven incremental updates. Explicitly modeling heterogeneity enables more accurate estimates of material flows, component-specific e-waste, and the effects of specialized AI hardware beyond traditional GPUs \citep{wang_2024_ewaste}.

\subsubsection{Workload-Specific Power Modeling}
AI workloads are diverse in nature and computational intensity depending on the task, such as training versus inference or domain-specific models like NLP and computer vision. This methodology differentiates power consumption profiles based on workload type and stage, incorporating reductions from efficient model compression, pruning, and optimization techniques that alleviate hardware demands without sacrificing accuracy. By integrating workload-level variability, this modeling avoids one-size-fits-all power assumptions, thus enhancing granularity for environmental impact analysis and forecasting the energy and e-waste implications of evolving AI applications and optimizations \citep{wang_2024_ewaste, you-2023}.

\subsubsection{Circular Economy Feedback Loops}
The model explicitly incorporates circular economy principles by simulating how recycling success rates (modeled as a Beta distribution from 70\% to 95\%) affect the demand for new hardware, closing the loop between materials recovered from end-of-life servers and newly manufactured components. It also accounts for geopolitical supply chain constraints, including rare earth element availability and trade-based technical barriers affecting component sourcing and regional deployment. Strategies such as lifespan extension through improved maintenance, module reuse by dismantling and repurposing critical components, and stepwise upgrading are modeled as they demonstrably reduce net e-waste generation. These feedback mechanisms imbue the model with strategic insight into how circular processes can mitigate e-waste growth amid the GAI boom \citep{wang_2024_ewaste}.

\subsubsection{Probabilistic Scenario Analysis}
To address uncertainty inherent in parameters like hardware failure rates, utilization, and adoption speeds, the enhanced methodology replaces deterministic scenario evaluation with Monte Carlo simulations using empirically grounded parameter distributions (e.g., Normal, Beta, Triangular). This probabilistic approach yields confidence intervals, revealing the range of possible e-waste outcomes and identifying sensitivities and risks. By embracing stochastic variability through repeated sampling, the model offers robust foresight for decision-making and policy formulation under uncertainty, improving upon previous single-value point estimates \citep{wang_2024_ewaste}.

\subsubsection{Regional Infrastructure Variations}
Recognizing geographic heterogeneity, the model incorporates differential regional standards in data center efficiency, hardware lifespans affected by climate and usage patterns, and variability in recycling infrastructure and regulatory environments. Server deployment proportions are modeled across major clusters—North America, East Asia, and Europe—with ±6 months timing variation to reflect logistic and supply chain disparities. This spatial differentiation is critical to accurately representing global e-waste flows and identifying region-specific challenges and interventions \citep{wang_2024_ewaste}.

\subsubsection{Lifecycle Assessment Enhancement}
Extending beyond servers, the methodology includes networking equipment, storage arrays, and cooling infrastructure in the lifecycle inventory, modeling component-level replacement patterns such as discrete GPU swaps versus full server decommissioning. Emissions from manufacturing and transportation phases and toxic material releases (lead, chromium, cadmium) from e-waste are integrated, augmenting environmental impact assessments. Economic valuation of recovered materials (\~\$70 billion potential globally) and environmental benefits of recycling are quantified, thus providing a holistic lifecycle perspective bridging physical material flows and sustainability outcomes \citep{wang_2024_ewaste}.

\section{Research Activities and Milestones}
Use a flowchart for the research activities and a Gantt chart for the research schedules. 

\section{Expected Results and Impact}
In this section, discussion will be on novel theories/findings/knowledge and the impact on society, nation and/or economy.

%References
\printbibliography

\end{document}

